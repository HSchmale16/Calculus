\documentclass[10pt,oneside,letter]{article}

\usepackage{tikz}
\usepackage{tikz-fct}
\usepackage{amsmath}
\usepackage{hyperref}
\usepackage[top=1in,bottom=1in,left=1in,right=1in]{geometry}
\usepackage{listings}
\usepackage{color}

\definecolor{dkgreen}{rgb}{0,0.6,0}
\definecolor{gray}{rgb}{0.5,0.5,0.5}
\definecolor{mauve}{rgb}{0.58,0,0.82}

\lstset{frame=tb,
    language=Java,
    aboveskip=3mm,
    belowskip=3mm,
    showstringspaces=false,
    columns=flexible,
    basicstyle={\small\ttfamily},
    numbers=left,
    numbersep=5pt,
    numberstyle=\tiny\color{gray},
    keywordstyle=\color{blue},
    commentstyle=\color{dkgreen},
    stringstyle=\color{mauve},
    breaklines=true,
    breakatwhitespace=true,
    tabsize=3
}

\begin{document}
\title{Applying Ahmdahl's Law to the Mandelbrot Fractal}
\author{Henry J Schmale}
\date{May 2015}
\maketitle

% Write the abstract
\begin{abstract}
The mandelbrot fractal is an incredibly time consuming calculation.
The drawing of the mandelbrot fractal can be speed up by multithreading.
However, there is a finate amount of performance that can be gained by
this method. The law governing this performance gain is called Amdahl's
Law.
\end{abstract}

\section{The Mandelbrot Set}

\section{Amdahl's Law}
Ahmdahl's Law is used to find the maximum expected improvement to an overall
system when only one part of the system is improved. This law is most often
used in parallel computing to predict the theoretical maximum speedup when
using many processors to operate on a workload in parallel. This law is named
after the computer architect Gene Amdahl.

The speedup of a program using many processors in parallel is limited by the
amount of time needed for the serial fraction of the program. That is the
section that must be executed in a single thread of execution, or the part
that cannot be parallelized.

\subsection{Equations}
Given:
\begin{itemize}
  \item $n \epsilon N$ The number of threads of execution.
  \item $B \epsilon [0,1]$ The fraction of the algorithim that is
      strictly serial.
\end{itemize}

$T(n)$ describes how long an algorithim will take to complete when running
with $n$ threads of execution, see equation \eqref{eq:amdahl_time}.
\begin{equation} \label{eq:amdahl_time}
    T(n) = T(1)(B + \frac{1}{n}(1-B))
\end{equation}

Therefore, the theoretical speedup $S(n)$ that can be had on a given algorithim
on a system capable of executing $n$ threads of execution is seen in equation
\eqref{eq:amdahl_speedup}
\begin{equation} \label{eq:amdahl_speedup}
    S(n) = \frac{T(1)}{T(n)}
         = \frac{T(1)}{T(1)(B+\frac{1}{n}(1-B))}
         = \frac{1}{B+\frac{1}{n}(1-B)}
\end{equation}

% Begin the next section that ties everything togather
\section{Tying Amdahl's Law and the Mandelbrot Fractal Togather}
The program I wrote to calculate the mandelbrot fractal uses a nested loop to
draw to the screen, and that has a time complexity of at least $O(n^2)$. That
part is strictly serial. Since the drawing to the screen routine is strictly
serial, amdahl's law applies. Now the worst case time complexity of calculating a
point in the mandelbrot fractal using the escape time algorithim $O({colors} + 1)$,
and the best case senario is $O(1)$. This leads to a worst case senario for the
whole program as $O(n^5)$.

Now plugging the numbers into amdahl's equations yeilds the following speed up
when running on 5 threads (4 worker threads and 1 main thread). THe speed up
yielded by multithreading can be seen in equation \eqref{eq::speedup5threads},
the final expected speed up running on 5 threads is 1.92 times faster than running
it in a single thread.
Given:
\begin{itemize}
    \item $T(1) = n^3$ The single threaded time complexity is $O(n^3)$
    \item $B = 2/5 = .4$ Roughly $2/5$ of the algorithim is serial
    \item The equations and givens in section~\pageref{sec:Ahmdahl's Law} still apply.
\end{itemize}

\begin{equation} \label{eq:speedup5threads}
    \begin{align*}
        T(n) = T(1)(B + \frac{1}{n}(1-B))\\
        T(5) = (n^3)(.4 + \frac{1}{5}(1-.4))\\
        T(5) = .52(n^3)\\
        S(5) = \frac{T(1)}{T(5)} = \frac{n^3}{.52n^3} = 1.923
    \end{align*}
\end{equation}

Additionally, there is scalling involved in this, as seen in the graphs below.

\begin{tikzpicture}
    \begin{axis}[
        xlabel=$number of tasks$,
        ylabel=$T(x)$
    ]
        \addplot gnuplot[id=t1]{x**3}
        \addplot gnuplot[id=t2]{x**3(.4 + (1/2)(.6))}
\end{tikzpicture}

% Distribute the source code in the handout
\section{The Source Code}
This section consists entirely of the source code to my multithreaded
mandelbrot fractal program written in C++. The entire contents of this project
is available at \url{<https://github.com/HSchmale16/CalculusProject>}.
\lstinputlisting[language=C++]{mandelbrot.cpp}

\end{document}
